\section{Conclusion}
DarkPatternMonitor provides a practical, benchmark-calibrated pipeline for detecting manipulation markers in real LLM chat logs. By combining DarkBench elicitation, LLM-judge labeling, and a scalable embedding classifier, the system enables prevalence measurement, gap analysis, and reliability auditing at operational scales. Reported findings indicate measurable manipulation markers in production logs, increased rates in longer conversations, and strong topic-dependent variability.

Based on our findings, we propose a \textbf{three-tier monitoring framework} for production deployment: (1) \textit{High-Alert} for character interaction and roleplay topics (5x baseline review rate), (2) \textit{Escalation-Watch} for user assistance conversations exceeding 10 turns, and (3) \textit{Standard} lightweight monitoring for technical and coding topics. This topic-aware approach enables efficient resource allocation for trust and safety teams.

\paragraph{Call for frontier model data.} Our methodology is model-agnostic, but current analysis is limited to GPT-3.5/GPT-4 conversations from 2023--2024. We urge AI labs and researchers to collaborate on creating a WildChat-equivalent dataset for frontier models (Claude, Gemini, o1) with opt-in consent and regular updates. Such a resource would enable the AI safety community to track manipulation patterns as model capabilities evolve and alignment techniques improve.
